\chapter{文本处理}
\label{chap:processing_words}

\marginpar{111}
本章的程序指向一个共同主题: 文本处理. 示例程序涵盖的范围包括随机单词
与句子的生成 (生成的句子可以和用户进行有限的对话), 以及文本处理. 大多数示例
程序都很简单, 它们只是起说明作用, 但是, 其中一些文档准备程序的确拥有实际
用途.

\section{随机文本发生器}
\label{sec:random_text_generation}

生成随机数据的程序有许多种用途. 这种程序可以用内建函数 \texttt{rand} 来
创建, 该函数每次被调用时, 都会返回一个伪随机数. \texttt{rand} 每次都使用
同一个种子数来生成随机数, 所以, 如果你想要得到一个不同的随机数序列, 你
就必须调用一次 \texttt{srand()}, 它根据当前时间计算出一个种子数, 并用
该种子数初始化 \texttt{rand}.

\subsection{随机选择}
\label{subsec:random_choices}

\texttt{rand} 每次被调用时都会返回 一个大于等于 0, 小于 1 的浮点数, 但是
一般来说, 更通常的需求是返回一个 \texttt{1} 到 \texttt{n} 之间的随机整数,
我们可以用 \texttt{rand} 来实现:
\begin{myverb}
    # randint - return random integer x, 1 <= x <= n

    function randint(n) {
        return int(n * rand()) + 1
    }
\end{myverb}
\texttt{randint(n)} 按比例调整 \texttt{rand} 的返回值, 调整后的值大于
等于 \texttt{0} 并且小于 \texttt{n}, 将小数部分截去可以得到 \texttt{0}
到 \texttt{n-1} 的整数, 然后再加 1, 就是 \texttt{1} 到 \texttt{n} 之间的
整数.

我们可以用 \texttt{randint} 来随机选择一个字母:
\marginpar{112}
\begin{myverb}
    # randlet - generate random lower-case letter

    function randlet() {
        return substr("abcdefghijklmnopqrstuvwxyz", randint(26), 1)
    }
\end{myverb}

利用 \texttt{randint}, 很容易就可以从一个 \texttt{n} 项的数组中随机选择一
个元素:
\begin{myverb}
    print x[randint(n)]
\end{myverb}
一个更有趣的问题是从某个数组中随机选择几项, 被选中的项必须按照原来的顺序
排列. 举例来说, 如果数组 \texttt{x} 按照升序排列, 则被选中的元素也要按照
升序排列.

函数 \texttt{choose} 从数组 \texttt{A} 的前 \texttt{n} 中随机选择
\texttt{k} 个元素, 并按照原来的顺序打印出来:
\begin{myverb}
    # choose - print in order k random elements from A[1]..A[n]

    function choose(A, k, n,    i) {
        for (i = 1; n > 0; i++)
            if (rand() < k/n--) {
                print A[i]
                k--
            }
    }
\end{myverb}
在函数内, \texttt{k} 是还需要打印的项的数目, \texttt{n} 是数组中等待检验
的元素个数. 打印第 \texttt{i} 个元素的条件是 \verb'rand() < k/n', 每当有
一个元素被打印出来, \texttt{k} 就递减一次, 每当判断条件 \verb'rand() < k/n'
被测试一次, \texttt{n} 就递减一次.

\begin{exercise}
    测试 \texttt{rand} 的输出是不是真的随机数.
\end{exercise}
\begin{exercise}
    写一个程序, 该程序生成 1 到 $n$ 之间的 $k$ 个不同的随机整数,
    程序的时间复杂 度与 $k$ 成正比.
\end{exercise}
\begin{exercise}
    写一个程序, 该程序生成随机的桥牌手 (bridge hands).
\end{exercise}

\subsection{废话生成器}
\label{subsec:cliche_generation}

我们的下一个例子是废话生成器 (cliche generator),
它根据已有的废话重新创建一个新的出来. 输入是一个句子集合:
\begin{myverb}
    A rolling stone:gathers no moss.
    History:repeats itself.
    He who lives by the sword:shall die by the sword.
    A jack of all trades:is master of none.
    Nature:abhors a vacuum.
    Every man:has a price.
    All's well that:ends well.
\end{myverb}
冒号将主语和谓语分开. 我们的废话生成器随机选择一个主语与另一个谓语作组合,
\marginpar{113}
如果运气好的话, 可能会产生很有意思的格言警句:
\begin{myverb}
    A rolling stone repeats itself.
    History abhors a vacuum.
    nature repeats itself.
    All's well that gathers no moss.
    He who lives by the sword has a price.
\end{myverb}
程序的代码实现非常明确:
\begin{myverb}
    # cliche - generate an endless stream of cliches
    #     input:  lines of form subject:predicate
    #     output: lines of random subject and random predicate

    BEGIN { FS = ":" }
          { x[NR] = $1; y[NR] = $2 }
    END   { for (;;) print x[randint(NR)], y[randint(NR)] }

    function randint(n) { return int(n * rand()) + 1 }
\end{myverb}
请注意, 程序中的死循环是有意为之.

\subsection{随机语句}
\label{subsec:random_sentences}

\cterm{上下文无关语法} (\term{contex-free grammar}) 指的是一组规则, 这组
规则定义了如何生成或分析一个语句集合. 每一条规则 (称为 \cterm{产生式}
(\term{production})) 都具有形式:
\begin{pattern}
    \indent\indent\textit{A} $\longrightarrow$ \textit{B C D} ...
\end{pattern}
该产生式的意思是每一个 \textit{A} 都可以被 ``重写'' 为 \textit{B C D ...}.
产生式左边的符号 (\textit{A}) 称为 \cterm{非终结符} (\term{nonterminal}),
它可以被进一步地扩展. 产生式右边的符号可以是非终结符 (可以是多个
\textit{A}) 或 \cterm{终结符} (\term{terminal}), 终结符指的是不能被扩展的
符号. 多个产生式可以共享同一个非终结符, 终结符与非终结符也可以在产生式的
右边出现多次.

我们将在第 \ref{chap:little_languages} 章展示 awk 的部分语法规则, 并利用
该规则开发一个语法分析器, 用来分析 awk 程序. 然而在这一章, 我们感兴趣的是
规则的生成, 而不是分析. 举例来说, 类似 ``the boy walks slowly'' 和 ``the
girl runs very very quickly'' 这样的句子可以用下面的语法来描述:
\begin{myverb}
    Sentence -> Nounphrase Verbphrase
    Nounphrase -> the boy
    Nounphrase -> the girl
    Verbphrase -> Verb Modlist Adverb
    Verb -> runs
    Verb -> walks
    Modlist -> very Modlist
    Adverb -> quickly
    Adverb -> slowly
\end{myverb}

\marginpar{114}
如下所示, 产生式为非终结符生成语句. 假设 \texttt{Sentence} 是起始非终结符,
那么选择一条以该符号作为左部的产生式:
\begin{myverb}
    Sentence -> Nounphrase Verbphrase
\end{myverb}
接下来, 从右部选择一个非终结符, 比如说 \texttt{Nounphrase}, 然后用以
\texttt{Nounphrase} 作为左部的产生式替换掉 \texttt{Nounphrase}:
\begin{myverb}
    Sentence -> Nounphrase Verbphrase
             -> the boy Verbphrase
\end{myverb}
不断地使用该方法, 直到所有的非终结符都被替换掉为止:
\begin{myverb}
    Sentence -> Nounphrase Verbphrase
             -> the boy Verbphrase
             -> the bov Verb Modlist Adverb
             -> the boy walks very Modlist Adverb
             -> the boy walks very Adverb
             -> the boy walks very quickly
\end{myverb}
\texttt{Sentence} 的最终展开结果是一个句子. 非终结符的推导过程与我们在初
级学校所学到的语句图 (sentence-diagram) 刚好相反: 我们现在是把动词短语拆
分成动词与副词, 而不是把动词与副词组合成动词短语.

\texttt{Modlist} 的产生式比较有趣. 一条规则是说用 \texttt{very Modlist}
替换 \texttt{Modlist}, 每次使用这条规则都会使句子变长. 幸运地是, 只要运用
另一条产生式规则 (该规则用空字符串替换掉 \texttt{Modlist}) 就可以终止潜在
的无限循环.

我们现在要开发一个程序, 该程序根据语法生成语句, 每次生成都从一个指定的非
终结符开始. 程序从文件中读取语法规则, 记录下每一个左部出现的次数, 
左部所拥有的右部的个数, 以及它们各自的组成成分. 然后, 每输入一个非
终结符, 就会为该非终结符生成一个随机语句.

程序使用三个数组来存放语法规则:
\texttt{lhs[A]} 给出了非终结符 \texttt{A} 的产 
生式个数, \texttt{rhscnt[A, i]} 存放的是 \texttt{A} 的第 \texttt{i} 条产生
式右部的符号个数, \texttt{rhslist[A, i, j]} 存放的是 \texttt{A} 的第 
\texttt{i} 条产生式右部的第 \texttt{j} 个符号. 对于前面提到的语法规则, 
三个数组的内容分别是:
\marginpar{115}
