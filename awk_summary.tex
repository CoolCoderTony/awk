% vim: ts=8 sts=8 sw=4 et tw=75
\chapter{AWK 总结}
\label{chap:awk_summary}

\marginpar{187}
这个附录包含了 awk 语言的一个总结. 在句法规则上, 如果某个成分被一对中括号
\verb'['...\verb']' 包围, 则表示它们是可选的.

\subsubsection{命令行}
\begin{quote}
    \texttt{awk [-F}\textit{s}\texttt{] '}%
    \textit{program}\texttt{'}\ \ \textit{optional list of filenames}

    \texttt{awk [-F}\textit{s}\texttt{] -f}
    \textit{progfile}\ \ \textit{optional list of filenames}
\end{quote}
参数 \texttt{-F}\textit{s} 把字段分隔符 \texttt{FS} 设置成 \textit{s},
如果没有提供文件名, awk 就从标准输入读取数据. 文件名的形式可以是 
\textit{var}\texttt{=}\textit{text}, 在这种情况下, 相当于把 \textit{text}
赋值给变量 \textit{var}, 当这个参数被当作一个文件而被访问时, 执行赋值
操作.

\subsubsection{AWK 程序}
一个 awk 程序由一系列的 \patact 语句和函数定义组成. 一个 \patact 语句具有
形式:
\begin{quote}
    \textit{pattern}\ \ \verb'{'\ \textit{action}\ \verb'}'
\end{quote}
如果某个动作省略了模式, 则默认匹配所有输入行; 如果某个模式省略了动作, 则
默认打印匹配行.

一个函数定义具有形式:
\begin{quote}
    \texttt{function}\ \
    \textit{name}\verb'('\textit{parameter-list}\verb') {'
        \textit{statement}\ \verb'}'
\end{quote}
\patact 语句和函数定义由换行符或分号分隔, 并且这两个字符可以混合使用.

\subsubsection{模式}
\begin{quote}
    \texttt{BEGIN}

    \texttt{END}

    \textit{expression}

    \verb'/'\textit{regular expression}\verb'/'

    \textit{pattern}\ \verb'&&'\ \textit{pattern}

    \textit{pattern}\ \verb'||'\ \textit{pattern}

    \verb'!'\textit{pattern}

    \verb'('\textit{pattern}\verb')'

    \textit{pattern}\verb','\ \textit{pattern}
\end{quote}

最后一个模式是范围模式, 它不能作为其他模式的组成部分. 类似地,
\texttt{BEGIN} 和 \texttt{END} 也不能和其他模式结合.
\marginpar{188}
\subsubsection{动作}
一个动作由一系列的语句组成, 这些语句包括:
\begin{quote}
    \texttt{break}

    \texttt{continue}

    \texttt{delete}\ \textit{array-element}

    \texttt{do}\ \textit{statement}\ \texttt{while}\
    \texttt{(}\textit{expression}\texttt{)}

    \texttt{exit[}\textit{expression}\texttt{]}

    \textit{expression}

    \texttt{if (}\textit{expression}\texttt{) }\textit{statement}
    \ \texttt{[else}\ \textit{statement}\texttt{]}

    \textit{input-output statement}

    \texttt{for (}\textit{expression}\texttt{; }\textit{expression}
    \texttt{; }\textit{expression}\texttt{) }\textit{statement}

    \texttt{for (}\textit{variable}\texttt{ in }\textit{array}\texttt{) }
    \textit{statement}

    \texttt{next}

    \texttt{return [}\textit{expression}\texttt{]}

    \texttt{while (}\textit{expression}\texttt{) statement}

    \verb'{' \textit{statement} \verb'}'
\end{quote}
一个单独的分号表示空语句. 在一个 \texttt{if-else} 语句中, 如果第一个 
\textit{statement} 和 \texttt{else} 出现在同一行, 那么它必须以分号结尾, 
或者用花括号包围起来. 类似地, 在 \texttt{do} 语句中, 如果
\textit{statement} 和 \texttt{while} 出现在同一行, 那么它必须以分号结尾,
或者用花括号包围起来.

\subsubsection{程序格式}
语句通过换行符或 (和) 分号隔开. 空行可以出现在任何语句, \patact 语句,
或函数定义的前面或后面. 空格与制表符可以插入到运算符或操作数的周围. 一条
长语句可以通过反斜杆延续到下一行. 另外, 如果一条语句在逗号, 左花括号,
\verb'&&', \verb'||', \texttt{do}, \texttt{else}, \texttt{if} 或 
\texttt{for} 的右括号后断行, 则不需要反斜杆. 由 \verb'#' 开始的注释可以
出现在任意一行的末尾.

\subsubsection{输入输出}
\begin{quote}
    \begin{tabbing}
        \texttt{close(}\textit{expr}\texttt{)}\hspace{8em} 
        \= 关闭由 \textit{expr}
        指示的文件或管道 \\
        \texttt{getline} \> 把 \verb'$0' 设置成下一条记录; 同时设置
        \texttt{NF}, \texttt{NR}, \texttt{FNR} \\

        \texttt{getline <}\textit{file} \> 把 \verb'$0' 设置成文件
        \textit{file} 的下一条记录; 同时设置 \texttt{NF} \\

        \texttt{getline}\ \textit{var} \> 把 \textit{var} 设置成下一条记录;
        同时设置 \texttt{NR}, \texttt{FNR} \\

        \texttt{getline}\ \textit{var}\ \texttt{<}\textit{file} \>  把
        \textit{var} 设置成文件  \textit{file} 的下一条记录. \\

        \texttt{print}  \> 打印当前记录 \\

        \texttt{print}\ \textit{expr-list} \> 打印 \textit{expr-list}
        所表示的表达式 \\

        \texttt{print}\ \textit{expr-list}\ \texttt{>}\textit{file} \>
        把表达式输出到文件 \textit{file} 中 \\

        \texttt{printf}\ \textit{fmt}\texttt{,} \textit{expr-list} \>
        格式化并输出 \\

        \texttt{printf}\ \textit{fmt}\texttt{,} \textit{expr-list}\ 
        \texttt{>} \textit{file}        \> 格式化并输出到文件 \textit{file}
        中 \\

        \texttt{system(}\textit{cmd-line}\texttt{)}     \> 执行命令
        \textit{cmd-line}, 返回命令的退出状态 \\
    \end{tabbing}
\end{quote}

\texttt{print} 后面的 \textit{expr-list}, 以及 \texttt{printf} 后面的
\textit{fmt}\texttt{,}\ \textit{expr-list} 可以用括号括起来. 在 
\texttt{print} 和 \texttt{printf} 中, \texttt{>>}\textit{file} 表示把
输出追加到文件 \textit{file} 的末尾, \texttt{|}\ \textit{command} 表示把
输出写到一个管道中. 类似的, \textit{command}\texttt{ | getline} 表示把
命令 \textit{command} 的输出以管道的方式输送给 \texttt{getline}. 函数
\texttt{getline} 在遇到文件末尾时返回 0, 出错时返回 -1.

\marginpar{189}

\subsubsection{\textbf{\texttt{printf}} 格式转换}
\texttt{printf} 与 \texttt{sprintf} 识别以下格式转换命令:
\begin{quote}
    \begin{tabbing}
        \verb'%c' \hspace{4em}      \= ASCII 字符 \\

        \verb'%d'       \> 十进制数 \\

        \verb'%e'       \> \texttt{[-]d.ddddddE[+-]dd}  \\

        \verb'%f'       \> \texttt{[-]ddd.dddddd} \\

        \verb'%g'       \> 效果等价于 \verb'%e' 或 \verb'%f', 选择转换
                后长度较短的那个, 无意义的零会被删除    \\

        \verb'%o'       \> 无符号八进制数 \\

        \verb'%s'       \> 字符串       \\

        \verb'%x'       \> 无符号十六进制数 \\

        \verb'%%'       \> 打印一个百分号 \verb'%', 不会有参数被转换 \\
    \end{tabbing}
\end{quote}

\verb'%' 与控制字符之间可以出现额外的参数:

\begin{quote}
    \begin{tabbing}
        \texttt{-} \hspace{4em}     \= 表达式在它所处的域中左对齐   \\

        \textit{width}  \> 当需要时, 把域的宽度填充到该值, 前导的
        \texttt{0} 表示用 \texttt{0} 填充 \\

        \texttt{.}\textit{prec} \> 字符串最大宽度, 或小数点后保留的位数 \\
    \end{tabbing}
\end{quote}

\subsubsection{内建变量}
下面列出的内建变量可以使用在任意一个表达式中:
\begin{quote}
    \begin{tabbing}
        \texttt{ARGC}\hspace{6em}   \= 命令行参数的个数 \\
        \texttt{ARGV}   \> 命令行参数组成的数组 (\texttt{ARGV[0..ARGC-1]})\\
        \texttt{FILENAME}\> 当前输入文件的文件名 \\
        \texttt{FNR}    \> 当前输入文件已读取的记录个数 \\
        \texttt{FS}     \> 输入数据的字段分隔符 (默认是空格) \\
        \texttt{NF}     \> 当前输入记录的字段个数 \\
        \texttt{NR}     \> 从程序开始到现在, 读取到的记录个数 \\
        \texttt{OFMT}   \> 数字的输出格式 (默认是 \verb'"%.6g"') \\
        \texttt{OFS}    \> 输出字段分隔符 (默认是空格)\\
        \texttt{ORS}    \> 输出记录分隔符 (默认是换行符) \\
        \texttt{RLENGTH}\> 被 \texttt{match} 中的正则表达式匹配的字符串
        的长度 \\
        \texttt{RS}     \> 输入数据的记录分隔符 (默认是换行符) \\
        \texttt{RSTART} \> 被 \texttt{match} 匹配的字符串在原字符串中的
        开始位置 \\
        \texttt{SUBSEP} \> 具有形式
        \texttt{[}\textit{i}\texttt{,}\textit{j}\texttt{,}...\texttt{]}
        的数组下标的分隔符 (默认是 \verb'"\034"') \\
    \end{tabbing}
\end{quote}
\texttt{ARGC} 和 \texttt{ARGV} 包含被执行的程序的名字 (通常是
\texttt{awk}), 但是不包括出现在命令行中的 awk 程序或选项.
\texttt{RLENGTH} 同时也是 \texttt{match} 的返回值.

\subsubsection{内建字符串函数}
在下面列出的字符串函数中, \textit{s} 和 \textit{t} 表示字符串, \textit{r}
表示正则表达式, \textit{i} 和 \textit{n} 表示整数.

\texttt{sub} 和 \texttt{gsub} 的替换字符串中的 \verb'&' 会被匹配的字符
串替换掉, 而 \verb'\&' 表示一个字面上的 \verb'&'.
\marginpar{190}
\begin{quote}
    \begin{tabular}{ll}
        \texttt{gsub(}\textit{r}\texttt{,} \textit{s}\texttt{,}
        \textit{t}\texttt{)} &
        \makecell[tl]{全局地替换 \textit{s} 中被 \textit{r} 匹配的 
        子字符串 \textit{t}, 返回替换发生的次数; \\ 如果省略 \textit{t},
        则默认使用 \texttt{\$0}} \\

        \texttt{index(}\textit{s}\texttt{,} \textit{t}\texttt{)} &
        返回 \textit{t} 在 \textit{s} 中的开始位置, 如果 \textit{s}
        不包含 \textit{t}, 则返回 0 \\

        \texttt{length(}\textit{s}\texttt{)} & 返回 \textit{s} 的长度 \\

        \texttt{match(}\textit{s}\texttt{,} \textit{r}\texttt{)} &
        \makecell[tl]{返回 \textit{s} 中匹配 \textit{r} 的子字符串的起始
        位置, 如果 \\ 不存在可匹配的子字符串, 则返回 0,
        调用该函数会同时设置 \\ \texttt{RSTART} 与 \texttt{RLENGTH}} \\

        \texttt{split(}\textit{s}\texttt{,} \textit{a}\texttt{,}
        \textit{fs}\texttt{)} & \makecell[tl]{按照 \textit{fs}, 把
        \textit{s} 切分到数组 \textit{a} 中, 返回 
        分割后的字段的个数;\\ 如果省略 \textit{fs}, 则默认使用
        \texttt{FS}} \\

        \texttt{sprintf(}\textit{fmt}\texttt{,} \textit{expr-list}
        \texttt{)} & 返回格式化了的 \textit{expr-list} (
        根据 \textit{fmt} 进行格式化) \\

        \texttt{sub(}\textit{r}\texttt{,} \textit{s}\texttt{,}
        \textit{t}\texttt{)}   & 类似于 \texttt{gsub}, 但是它只替换
        第一个被匹配的子字符串 \\

        \texttt{substr(}\textit{s}\texttt{,} \textit{i}\texttt{,}
        \textit{n}\texttt{)} & \makecell[tl]{返回 \textit{s} 中, 从
        \textit{i} 开始的, 长度为 \textit{n} 的子字符串, \\ 
        如果省略 \textit{n}, 则返回
        \textit{s} 中从 \textit{i} 开始的后缀} \\
    \end{tabular}
\end{quote}

\subsubsection{内建算术函数}
\begin{quote}
    \begin{tabbing}
        \texttt{atan2(}\textit{y}\texttt{,} \textit{x}\texttt{)}
        \hspace{4em} \=
        \textit{y}/\textit{x} 的反正切值, 弧度制,
        定义域从 $-\pi$ 到 $\pi$ \\

        \texttt{cos(}\textit{x}\texttt{)} \> 余弦 (弧度制) \\

        \texttt{exp(}\textit{x}\texttt{)} \> 指数 $e^x$ \\

        \texttt{int(}\textit{x}\texttt{)} \> 取整 \\

        \texttt{log(}\textit{x}\texttt{)} \> 自然对数 \\

        \texttt{rand(}\texttt{)} \> 返回一个伪随机数 \textit{r}
        (0 $\leqslant$ \textit{r} $<$ 1 )\\

        \texttt{sin(}\textit{x}\texttt{)} \> 正弦 (弧度制) \\

        \texttt{sqrt(}\textit{x}\texttt{)} \> 方根 \\

        \texttt{srand(}\textit{x}\texttt{)} \> 设置随机数种子, 如果省略
        \textit{x}, 则默认使用
        % FIXME: 脚注没显示出现
        当天的时间 \footnote{原文是 uses time of day if no \textit{x} given}
    \end{tabbing}
\end{quote}
\subsubsection{表示式运算符 (按优先级递增排列)}
表达式可以通过下列运算符进行组合:
\begin{quote}
    \begin{tabbing}
        \verb'= += -= *= /= %= ^=' \hspace{2em} \= 赋值 \\

        \texttt{?:} \> 条件表达式 \\

        \texttt{||} \> 逻辑或 \\

        \verb'&&' \> 逻辑与 \\

        \texttt{in} \> 数组成员运算符 \\

        \verb'~  !~' \> 正则表达式匹配运算符, 与否定匹配运算符 \\

        \texttt{< <= > >= != ==} \> 关系运算符 \\

        \texttt{ }      \> 字符串拼接 (没有显式的运算符) \\

        \texttt{+ -}    \> 加, 减 \\

        \verb'* / %'  \> 乘, 除, 取模 \\

        \texttt{+ - !}  \> 单目加, 单目减, 逻辑非 \\

        \texttt{\^}      \> 指数运算符 \\

        \texttt{++ --}  \> 自增, 自减 (包括前缀形式与后缀形式) \\

        \texttt{\$}     \> 字段 \\
    \end{tabbing}
\end{quote}
所有的运算符都是左结合的, 除了赋值运算符, \texttt{?:} 和 \texttt{\^},
它们是右结合的. 任意一个表达式都可以用括号括起来.
\marginpar{191}
\subsubsection{正则表达式}
正则表达式的元字符包括:
\begin{awkcode}
    \ ^ $ . [ ] | ( ) * + ?
\end{awkcode}
下面的表格总结了正则表达式及其所匹配的字符串:
\begin{quote}
    \begin{tabular}{ll}
        \textit{c}       & 匹配一个非元字符 \textit{c} \\
        \texttt{\textbackslash}\textit{c} & 匹配一个转义序列,
        或一个字面上的字符 \textit{c} \\
        \texttt{\^}     & 匹配一个字符串的开始 \\
        \texttt{\$}     & 匹配一个字符串的结束 \\
        \texttt{.}      & 匹配任意一个字符 \\
        \texttt{[\^}\textit{abc...}\texttt{]} & 字符集合: 匹配
        \textit{abc...} 中的任意一
        个字符 \\
        \textit{$r_1$}\texttt{|}\textit{$r_2$} & 选择: 匹配一个能被 \textit{$r_1$} 或 \textit{$r_2$} 匹配的字符串 \\
        \texttt{(}\textit{$r_1$}\texttt{)}\texttt{(}\textit{$r_2$}\texttt{)} & 拼接: 匹配字符串
        \textit{xy}, 其中 \textit{x} 被 \textit{$r_1$} 匹配, \textit{y} 被 \textit{$r_2$} 匹配 \\
        \texttt{(}\textit{r}\texttt{)*} & 匹配 0 个或多个连续出现的被 \textit{r} 匹配 
        的字符串 \\
        \texttt{(}\textit{r}\texttt{)+} & 匹配 1 个或多个连续出现的被 \textit{r} 匹配 
        的字符串 \\
        \texttt{(}\textit{r}\texttt{)?} & 匹配 0 个或 1 个被 \textit{r} 匹配的字符串 \\
        \texttt{(}\textit{r}\texttt{)}   & 组合: 匹配的字符串与 \textit{r} 所匹配 
        的字符串相同 \\

    \end{tabular}
\end{quote}
运算符按优先级升序排列. 只要没有违反优先级规则, 就可以省略多余的括号.

\subsubsection{转义序列}
在字符串与正则表达式中, 转义序列具有特殊的含义.
\begin{quote}
    \begin{tabbing}
        \verb'\b' \hspace{4em} \= 退格 \\
        \verb'\f' \> 换页 \\
        \verb'\n' \> 换行 \\
        \verb'\r' \> 回车 \\
        \verb'\t' \> 制表 \\
        \verb'\'\textit{ddd} \> 八进制数, \textit{ddd} 是 1 到 3 个数字,
        每个数字的值在 0 到 7 之间 \\
        \verb'\'\textit{c} \> 其他字面上的字符 \textit{c}, 比如 \verb'\"'
        表示 \verb'"', \verb'\\' 表示 \verb'\' \\
    \end{tabbing}
\end{quote}

\subsubsection{限制}
任意一个特定的 awk 实现都会强加一些限制条件, 下面列出了一些典型值:
\begin{quote}
    100 个字段 \par
    每条输入记录 3000 个字符 \par
    每条输出记录 3000 个字符 \par
    每个字段 1024 个字符 \par
    每个 \texttt{printf} 字符串 3000 个字符 \par
    字面字符串 400 个字符 \par
    字符集合 400 个字符 \par
    15 个打开文件 \par
    1 个管道 \par
    双精度浮点数
\end{quote}
数值的限制与本地系统所能表示的数值范围有关, 比如某个机器所能表示的数值 
范围是 $10^{-38}$ 到  $10^{38}$, 超过这个范围的数值只拥有字符串形式.
\marginpar{192}
\subsubsection{初始化, 比较和强制类型转换}
每一个变量或字段, 在任意时刻都可能是字符串, 或数值, 或两者都是. 当变量
通过赋值语句来获取一个值时:
\begin{awkcode}
    var = expr
\end{awkcode}
它的类型也会被设置成表达式的类型 (``赋值'' 包括 \texttt{+=}, \texttt{-=},
等等). 算术表达式的类型是数值, 拼接是字符串, 以此类推. 如果赋值语句 
只是一个简单的复制, 比如 \texttt{v1 = v2}, 那么 \texttt{v1} 就会被
设置成 \texttt{v2} 的类型.

比较时, 如果两个操作数的类型都是数值, 那么比较操作就会按照数值比较来进行.
否则的话, 操作数被强制转换成字符串 (如果原来不是字符串), 此时比较操作
就按照字符串比较来进行. 通过某些手段, 可以把任意一个表达式强制转换成数
值类型, 比如 
\begin{awkcode}
    expr + 0
\end{awkcode}
转换成字符串可以这样做 (也就是和空字符作拼接操作):
\begin{awkcode}
    expr ""
\end{awkcode}
字符串的数值形式的值, 指的是该字符串的数值前缀转换成数值后所得到的值.

未初始化的变量的值是 数值0 或空字符串 \verb'""'. 因此, 如果 \texttt{x}
没有被初始化过, 则条件判断
\begin{awkcode}
    if (x) ...
\end{awkcode}
就会失败, 同样, 下面这些条件判断:
\begin{awkcode}
    if (!x) ...
    if (x == 0) ...
    if (x == "") ...
\end{awkcode}
都会成功, 但是注意:
\begin{awkcode}
    if (x == "0") ...
\end{awkcode}
的比较结果为假.

如果可能的话, 字段的类型可以通过上下文环境来判断, 比如,
\begin{awkcode}
    $1++
\end{awkcode}
该表达式意味着: 如果有必要, 就把 \texttt{\$1} 强制转换成数值类型, 而
\begin{awkcode}
    $1 = $1 "," $2
\end{awkcode}
意味着: 如果有必要, 就把 \texttt{\$1} 和 \texttt{\$2} 的类型强制转换成
字符串.

如果无法根据上下文来判断变量的类型, 比如,
\begin{awkcode}
    if ($1 == $2) ...
\end{awkcode}
这时候就得根据输入数据来决定变量的类型. 所有字段的类型都是字符串, 但是,
如果字段包含了一个机器可识别的数, 那么它也会被当作数值类型.

显式为空的字段具有字符串值 \verb'""', 它们不是数值类型. 该结论也适用于
不存在的字段 (也就是超出 \texttt{NF} 的部分) 和 空白行的 \texttt{\$0}.

对字段成立的结论, 同样适用于由 \texttt{split} 创建的数组元素.

当在表达式中提到一个不存在的变量时, 就会创建该变量, 其初始值是 0 和
\texttt{""}. 因此, 如果元素 \texttt{arr[i]} 当前不存在, 语句:
\begin{awkcode}
    if (arr[i] == "") ...
\end{awkcode}
就会导致元素 \texttt{arr[i]} 被创建, 且初始值 为 \texttt{""}, 这就使得
\texttt{if} 的条件判断结果为真. 测试语句
\begin{awkcode}
    if (i in arr) ...
\end{awkcode}
判断元素 \texttt{arr[i]} 是否存在, 但是不会带来创建新元素的副作用.
